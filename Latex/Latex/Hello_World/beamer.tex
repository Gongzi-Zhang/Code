\documentclass[11pt,pdf,aspectratio=43]{beamer}
% aspectratio: set the aspect ration and the paper size.
% \mode<presentation>
% \usepackage{pgfpages}	
% \setbeameroption{show notes on second screen}	% need pgfpages package
\usetheme{Madrid}
\usefonttheme[onlylarge]{structuresmallcapsserif}
\usefonttheme[onlysmall]{structurebold}
\usecolortheme{orchid}
\setbeamercolor{frametitle}{bg=blue!30!black!80}
%\mode<presentation>{%
%    \usetheme[hideothersubsections, right,width=22mm]{Goettingen}
%    \usetheme{boxes}
%}
% \mode<presentation>{\usetheme{Warsaw}}
% General theme of presentation:
%	default	    Antibes	Berlin	    Copenhagen
%	Madrid	    Montpelier	Ilmenau	    Malmoe
%	Cambridge   Berkeley	Singapore   Warsaw
% Color themes:	beetle,	beaver,	orchid,	whale,	dolphin
% Inner themes:	circles,    rectanges,	rounded,    inmargin	% inner element: usually blocks
% Outer themes:	infolines,  smoothbars,	sidebar,    split,  tree    % headline, footline, and sidebar
%  Font thems:	serif,	structurebold,	structureitalicserif,	structuresmallcapseserif

\usepackage{fancybox}	%   to add fancy borders
\usepackage{listings}
\lstset{
    breaklines	= true,
    breakatwhitespace	= true,
}
\graphicspath{{figures/}}

\setbeamertemplate{navigation symbols}{}

%%%%% footline

% \setbeamertemplate{footline}[frame number]

% \setbeamertemplate{footline}{
%     \begin{beamercolorbox}[sep=1ex]{author in head/foot}
% 	\rlap{\textit{weekly meeting}}\hfill weibin zhang\hfill\llap{\insertframenumber}
%     \end{beamercolorbox}
% }

% \setbeamertemplate{footline}{
%     \hbox{
%     \begin{beamercolorbox}[wd=0.5\paperwidth,ht=2.5ex,dp=1.125ex,leftskip=.3cm plus1fill,rightskip=.3cm]{title in head/foot}
% 	\usebeamerfont{title in head/foot}\insertshorttitle
%     \end{beamercolorbox}
%     \begin{beamercolorbox}[wd=0.5\paperwidth,ht=2.5ex,dp=1.125ex,leftskip=.3cm,rightskip=.3cm plus1fill]{title in head/foot}
% 	\usebeamerfont{title in head/foot}\insertframenumber
%     \end{beamercolorbox}
%     }
%     \vskip0pt
% }

% title
\setbeamertemplate{frametitle}[default][center]
\setbeamercolor{frametitle}{fg=black}
\setbeamerfont{title}{shape=\itshape, family=\rmfamily}

\setbeamersize{text margin left=1cm, text margin right=1cm}
% \setbeamercolor{structure}{fg=craneblue}

% list
% \setbeamertemplate{itemize item}{X}

\title{Simple Presentation}
\subtitle{Beamer}   % optional
\author{Weibin Zhang}
\institute{
    \begin{tabular}{r@{ }l}
	1:& Univ. of Sci. and Tech. of China \\
	2:& Stony Brook University
    \end{tabular} 
}
\titlegraphic{\includegraphics[width=20mm]{SBU-circle}}
\date{\today}
\logo{\includegraphics[scale=0.05]{SBU-circle}}

% \definecolor{uofsgreen}{rgb}{.125,.5,.25}
% \usecolortheme[named=uofsgreen]{structure}

\AtBeginSection[]   % have this if you'd like a recurring outline
{
    \begin{frame}<beamer>
	\frametitle{Outline}
	\tableofcontents[currentsection]    % show ToC and highlight current section
    \end{frame}
}

\begin{document}
%\beamerdefaultoverlayspecification{<+->}   % If you always want to pause after items.

% \includeonlyframes{test1,test2}   % a comma-separated lsit (without spaces) of the names of frames.
%%%%%%%%%%%%%%%%%%%%%%%%%%%%%%%%%%%%%%%%%%%%
\begin{frame}
    \titlepage
\end{frame}

%%%%%%%%%%%%%%%%%%%%%%%%%%%%%%%%%%%%%%%%%%%%
\section[Frame]{What's a Frame}
% \section{A very long section name\breakhere second line}
% \section{A very long section name\protect\\ second line}
% \section{A very long section name\newline second line}

\begin{frame}
    \frametitle{Frame Components}

    \begin{enumerate}
	\item headline and footline
	\item left and right sidebar
	\item navigation bars
	\item navigation symbols
	\item logo
	\item frametitle [framesubtitle]
	\item background
	    \begin{itemize}
		\item background canvas
		\item main background
	    \end{itemize}
	\item contents
    \end{enumerate}

%    \note[item] {Hello World}
%    \note[item] {No longer than 2 mins}
\end{frame}

\begin{frame}[fragile]
    \frametitle{Headline}

    \begin{lstlisting}[language=TeX]
\setbeamertemplate{headline}
{
    \begin{beamercolorbox}{section in head/foot}
	\vskip2pt\insertnavigation{\paperwidth} \vskip2pt
    \end{beamercolorbox}
}
    \end{lstlisting}
\end{frame}

\section{Themes}
\label{sec:themes}
\begin{frame}[allowframebreaks]
    \frametitle{Themes in Beamer}
    \begin{itemize}
	\item No Navigation bar: default, boxes, Bergen, Pittsburgh and Rochester.
	\item Top bar:	Antibes, Darmstadt, Frankfurt, JuanLesPins, Montpellier and Singapore.
	\item Bottom bar: AnnArbor, Berlin, CambridgeUS, Copenhagen, Dresden, Ilmenau, Luebeck, Malmoe, Szeged and Warsaw.
	\item Side Bar: Berkeley, Goettingen, Hannover, Marburg and PaloAlto.
    \end{itemize}

    Actually, each themes is divided into four subthemes: Outer theme, Inner theme, Color theme and Font theme.
    \begin{itemize}
	\item Outer theme: set top bar, bottom bar and side bar and their constructure. Set with \textbackslash{useoutertheme\{outertheme\}}, 
	    possible value include: \\
	    default, infolines, miniframes, sidebar, smoothbars, split, shadow, tree and smoothtree.
	\item Inner theme: set the layout of main content(Headline, Table, Theorem etc.). Set with \textbackslash{useinnertheme\{innertheme\}},
	    possible value include: \\
	    default, circles, rectangles and rounded.
	\item Color theme: set color configuration. set with \textbackslash{usecolorthem}\{colortheme\}, possible value include: \\
	    default, albatross, beaver, beetle, crane, dolphin, dove, fly, lily, orchid, rose, seagull, seahorse, sidebartab, 
	    structure, whale and wolverine.
	\item Font theme. Set with \textbackslash{usefonttheme}\{fonttheme\}, possible value is: \\
	    default, serif, structurebold, structureitalicserif and structuresmallcapsserif.
    \end{itemize}
\end{frame}

\begin{frame}[fragile]
    \frametitle{mode}

    \begin{lstlisting}[language=TeX]
\mode<mode>
    \end{lstlisting}
    \begin{itemize}
	\item article
	\item presentation
	    \begin{itemize}
		\item beamer
		\item handout
		\item trans
	    \end{itemize}
    \end{itemize}
\end{frame}

\section{Basic Ideas}
\label{sec:basic ideas}

%%%%%%%%%%%%%%%%%%%%%%%%%%%%%%%%%%%%%%%%%%%%
\subsection*{Adding Slides}
\newenvironment{slide}
    {\begin{frame}[fragile,environment=slide]}
    {\end{frame}}

\begin{slide}
    \frametitle{Adding slides}
    \begin{itemize}
	\item Create a slide use the following code: \\
	\begin{lstlisting}[language=TeX]
\begin{frame}[options]
    \frametitle{title}
    \framesubtitle{subtitle} "optional" 
	<contents> 
\end{frame}
	\end{lstlisting}
	\item The possible options include: \\
	    \begin{itemize}
		\item \textbf{plain} removes all slide decorations ( useful for larger images and first slide )
		\item \textbf{c and b} align contents of the slide in the middle or bottom (default alignment is top,
		    but this can easily be changed in the document class options)
		\item \textbf{fragile} is necessary for slides that use the verbatim
		\item \textbf{shrink} automagically makes the contents fit on one slide
		\item \textbf{allowframebreaks} splits contents of a frame if it does not fit.
	    \end{itemize}
    \end{itemize}
\end{slide}

%%%%%%%%%%%%%%%%%%%%%%%%%%%%%%%%%%%%%%%%%%%%
\begin{frame}[c]    % vertical options: t--top; c--center; b--bottom;
    \frametitle{Title}
    \framesubtitle{Subtitle}
    \centering
    Title Page.
\end{frame}

%%%%%%%%%%%%%%%%%%%%%%%%%%%%%%%%%%%%%%%%%%%%
\subsection*{Table of Contents}
\begin{frame}
    \frametitle{Table of Contents}
    \begin{itemize}
	\item Create outline using \textbackslash{}tableofcontents[options ]
	\item The possible options:
	    \begin{itemize}
		\item	\textbf{currentsection} (all sections but current are greyed out)
		\item	\textbf{currentsubsection} ( all subsection but current are greyed out)
		\item	\textbf{hideallsubsections} (all subsections are hidden)
		\item	\textbf{hideothersubsections} (all subsections of sections other than the current are hidden)
		\item	\textbf{pausesections} (shows the table of contents incrementally)
		\item	\textbf{pausesubsections} (finer increments than \textbackslash{}pausesections)
		\item	\textbf{sections=\{2-3\}} (only section 2 and 3 displayed)
		\item	\textbf{sectionstyle=1/2}  
			(define style of current section (1), 
			other sections (2) using show, shaded and hide, e.g. sectionstyle=shaded/show)
		\item	\textbf{subsectionstyle=1/2/3}  
			(define style for current subsection(1), 
			other subsections in current section(2), 
			subsections in other sections(3))   
	    \end{itemize}
	\item The commands \textbackslash{}section, \textbackslash{}subsection, etc. make a struture for
	      tables of contents (outline are independent of slide titles)
    \end{itemize}
\end{frame}

%%%%%%%%%%%%%%%%%%%%%%%%%%%%%%%%%%%%%%%%%%%%
\subsection*{Fonts}
\begin{frame}
    \frametitle{Fonts}
    \begin{itemize}
	\item There are five font themes:
	    \begin{itemize}
		\item \textbf{default} (sans serif)
		\item \textbf{serif}
		\item \textbf{structurebold} (titles, headlines, etc. are typeset in a bold font)
		\item \textbf{structureitalicserif} (titles, headlines, etc. are typeset in an italic serif font)
		\item \textbf{structuresmallcapsserif} (titles, headlines, etc. are typeset in a small caps serif font)
	    \end{itemize}
	\item Change the document-wise font size to 10, 11(default), 12 points in the options of documentclass.
	\item Colour text using \textbackslash{}textcolor\{colour\}\{text\}
	\item \textbackslash{}alert\{text\} command colours text red
    \end{itemize}

    Two extra font sizes defined within {\bf beamer}:
    \begin{itemize}
	\item \textbackslash{Tiny}
	\item \textbackslash{TINY}
    \end{itemize}
\end{frame}

%%%%%%%%%%%%%%%%%%%%%%%%%%%%%%%%%%%%%%%%%%%%
\subsection*{semiverbatim}
\begin{frame}
    \frametitle{semiverbatim Environment}
    semiverbatim works like verbatim except that \fbox{\textbackslash}, \fbox{\{} and \fbox{\}} retain their meaning. \\
    This allows you to access Beamer formatting commands. If you want the command or environment to be
    ignored, you simply put a \fbox{\textbackslash} in front of it.

    NOTE: to use verbatim environment within a frame, be sure to declare the frame \emph{fragile}.

    \color{brown}   Example \\
    Using the \textbf{semiverbatim} environment, you can still \alert{format} verbatim text with Beamer
    commands or you can display commands \textbackslash{}alert\{like this\}
\end{frame}

%%%%%%%%%%%%%%%%%%%%%%%%%%%%%%%%%%%%%%%%%%%%
\subsection*{Spacing}
\begin{frame}
    \frametitle{Spacing}
    \begin{itemize}
	\item	Vertical space:	\textbackslash{vskip}[\emph{N}][unit]:	\textbackslash{vskip15pt}
	\item	Horizontal space: \textbackslash{hskip}[\emph{N}][unit]
	\item	Negative valuse can also be used to squeeze text or graphics together.
    \end{itemize}
\end{frame}

%%%%%%%%%%%%%%%%%%%%%%%%%%%%%%%%%%%%%%%%%%%%
\subsection*{Overlay}
\begin{frame}
    \frametitle{Overlay}
    There maybe more than one slide in some frames, which is called overlays.
    It come into effect by specifying the slides using \textless and \textgreater, or some special command: \textbackslash{pause}. \\
    Some commands have special overlay specification effects:	\\
    \begin{table}
	\scriptsize
	\begin{tabular}{c | p{0.75\textwidth} }
	    \hline
	    \textbackslash{onslide}\textless\textgreater	&   Text only appears on specified slides. If no text is given,
		text following the command will only appear on the specified slides.	\\
	    \hline
	    \textbackslash{only}\textless\textgreater\{\}	&   Text only appears on specified slides. When the text is hidden, it will
		occupy no space. \\ % corresponding environment: onlyenv
	    \hline
	    \textbackslash{visible}\textless\textgreater	&   Text only appears on specified slides and is completely transparent, 
		but still occupies space. \\	% corresponding environment: visibleenv
	    \hline
	    \textbackslash{invisible}\textless\textgreater	&   the opposite of visible.	\\  % invisibleenv
	    \hline
	    \textbackslash{alt}\textless\textgreater\{\}\{\}	&   Takes two arguments: one for the default text and a 
		second for the alternate text. The default text shows up on the specified slides. The alternate 
		text shows up on all unspecified slides.    \\	% altenv
	    \hline
	    \textbackslash{temporal}\textless\textgreater\{\}\{\}\{\} &   Takes three arguments: one for the text that
		will appear if the current slide comes before the specified slides, a second for the text that
		appears while currently on the specified slides, and a third for the text that appears after the
		specified slides have appeared.	\\
	    \hline
	    \textbackslash{uncover}\textless\textgreater\{\}	&   The text will only be ``uncovered" on the specified 
		slides. On other slides, the text will still be typeset and will appear transparent. \\	    % uncoverenv
	    \hline
	\end{tabular}
    \end{table}
\end{frame}

\begin{frame}{Other control commands}
    \begin{itemize}
	\item \textbackslash{}textbf\textless\textgreater\{\}
	\item \textbackslash{}textit\textless\textgreater\{\}
	\item \textbackslash{}color\textless\textgreater[]\{\}
	\item \textbackslash{}alert\textless\textgreater\{\}
	\item \textbackslash{}item\textless\textgreater
    \end{itemize}
\end{frame}
\begin{frame}{Overlay example}
    \onslide<1,2>{onslide}

    \only<1>{only}

    \visible<3,4>{visible}

    \invisible<2>{invisible}

    \alt<1>{default}{alternate}	

    \temporal<2>{fisrt}{second}{third}    

    \uncover<2,3>{uncover}

    \only<1-3,5> {specification style}
\end{frame}

%%%%%%%%%%%%%%%%%%%%%%%%%%%%%%%%%%%%%%%%%%%%
\subsection*{Columns}
\begin{frame}[fragile, label=columns]
    \frametitle{Columns}
    call column environment: \\
    \begin{lstlisting}[language=TeX]
\begin{columns}
    \column{.5\textwidth}
	First column text and/or code

    \column{.5\textwidth}
	Second column text and/or code
\end{columns}
    \end{lstlisting}
\end{frame}

%%%%%%%%%%%%%%%%%%%%%%%%%%%%%%%%%%%%%%%%%%%%
\subsection*{Blocks}
\begin{frame}
    \frametitle{Blocks}
    Blocks can be used to separate a specific section of text or graphics from the rest of the frame: \\
    \begin{block}{Blocks in Beamer}
	``Beamer is a \LaTeX{} class for creating presentations that are held using a projector \ldots "
    \end{block}
    Other block environments:
    \begin{block}{Other Block Environments}
	\begin{tabular}{c|c}
	    \textbf{Content Type}   & \textbf{Corresponding Environment}    \\
	    \hline
	    Generic &	block	\\
	    Theorems	&   theorem \\
	    Lemmas  &	lemma	\\
	    Proofs  &	proof	\\
	    Corollaries	&   corollary	\\
	    Examples	&   example	\\
	    Hilighted Title &	alertblock  \\  
	    \hline
	\end{tabular}
    \end{block}
\end{frame}

%%%%%%%%%%%%%%%%%%%%%%%%%%%%%%%%%%%%%%%%%%%%
\subsection*{Text Boxes}
\begin{frame}
    Use package `fancybox' to add fancy borders to your presentation.   \\
    \begin{block}{Text Border Examples}
	\shadowbox{shadow}
	\fbox{flat box}
	\doublebox{double box}
	\ovalbox{oval box}
	\Ovalbox{Oval box}
    \end{block}
\end{frame}

%%%%%%%%%%%%%%%%%%%%%%%%%%%%%%%%%%%%%%%%%%%%
\subsection*{Graphics}
\begin{frame}[fragile]
    \frametitle{Graphics}

    Sometimes you want to explain great details of a complex graphic, in
    this case, you can use \verb|\framezoom| to create anticipated 
    zoomings of interesting parts of the graphics.

\end{frame}

\subsection*{Transitions}
\begin{frame}{allowframebreaks}
    \frametitle{Transitions}
    A slide transition is composed of a single command. Slide transitions are overlay specification aware.
    And there are two possible options for each transition:
    \begin{itemize}
	\item duration=\textless seconds\textgreater specifies the number of seconds the transitions effect needs
	\item direction=\textless degree\textgreater specifies the direction for directed effects.
    \end{itemize}

    \begin{block}{Available Transitions}
	\small
	\begin{tabular}{rl}
	    \textbackslash{transblindshorizontal}   &	\textbf{Horizontal blinds pulled away}	\\
	    \textbackslash{transblindsvertical}	    &	\textbf{Vertical blinds pulled away}	\\
	    \textbackslash{transboxin}		    &	\textbf{Move to center from all sides}	\\
	    \textbackslash{transboxout}		    &	\textbf{Move to all sides from center}	\\
	    \textbackslash{transdissolve}	    &	\textbf{Slowly dissolve what was shown before}	\\
	    \textbackslash{transglitter}	    &	\textbf{Glitter sweeps in specified direction}	\\
	    \textbackslash{transslipverticalin}	    &	\textbf{Sweeps two vertical lines in}	\\
	    \textbackslash{transslipverticalout}    &	\textbf{Sweeps two vertical lines out}	\\
	    \textbackslash{transsliphorizontalin}   &	\textbf{Sweeps two horizontal lines in}	\\
	    \textbackslash{transsliphorizontalout}  &	\textbf{Sweeps two horizontal lines out}\\
	    \textbackslash{transwipe}		    &	\textbf{Sweeps single line in specified direction}  \\
	    \textbackslash{transduration\{2\}}	    &	\textbf{Show slide specified number of seconds}	
	\end{tabular}
    \end{block}
\end{frame}

\begin{frame}
    \frametitle{Transitions (continue)}
    \centering
    \transboxin
    Hello world
\end{frame}

%%%%%%%%%%%%%%%%%%%%%%%%%%%%%%%%%%%%%%%%%%%%%%%%%%%%%%%
\section{Blocks}
\label{sec:blocks}

\subsection*{Theorem}
\begin{frame}
    \frametitle{Theorem}
    \begin{Theorem}<1->
	There is no largest prime number.
    \end{Theorem}

    \begin{proof}<2->
	\begin{itemize}[<+->]
	    \item Assume there is largest prime number: \textbf{p}
	    \item let \textbf{N} be the product of all prime numbers $\le$ \textbf{p} plus 1:
		\[
		    N = 2 \times 3 \times 5 \times \cdots \times p + 1
		    \]
	    \item So \textbf{N} is larger than \textbf{p} and it can't be divided 
		    by any prime number $\le$ \textbf{p}, so it should be
		    a prime number; which conflicts with the assumption.
	\end{itemize}
    \end{proof}
\end{frame}


\appendix
%%%%%%%%%%%%%%%%%%%%%%%%%%%%%%%%%%%%%%%%%%%%%%%%%%%%%%%
\section{Overlay}
\label{sec:overlay}

\begin{frame}
    \frametitle{Overlay using \textbackslash{}pause}

    \begin{itemize}
	\item First \pause
	\item Second \pause
	\item Third
    \end{itemize}

    \onslide<4->\hyperlink{columns<1>}
    {\beamergotobutton{Go To Columns Slide}}
\end{frame}

\begin{frame}
    \frametitle{Overlays in List Environment}
    \begin{itemize}
	\item<1> On slide 1 only
	\item<2,4> On slides 2 and 4
	\item<3-> On slides from 3 on
	\item<4> On slide 4 only
	\item<-4> slides up to slide 4
	\item<5> On slide 5
    \end{itemize}
\end{frame}

\setbeamercovered{transparent}	% show covered parts in transparent style

\begin{frame}[<+->]
    \frametitle{Overlays using option to frame}
    \begin{itemize}
	\item First 
	\item Second
	\item Third
    \end{itemize}
\end{frame}

\setbeamercovered{transparent=70}   % set transparent to 70%

\begin{frame}
    \frametitle{Overlays using option to itemize}
    \begin{itemize}[<+->]
	\item First 
	\item Second
	\item Third
    \end{itemize}
\end{frame}

\begin{frame}
    \frametitle{Overlays using \textbackslash{}item specification}
    \begin{itemize}
	\item<+-> First 
	\item<+-> Second
	\item<+-> Third
    \end{itemize}
\end{frame}

\begin{frame}
    \frametitle{Overlays with alert}
    \begin{itemize}[<+-|alert@+>]
	\item First
	\item Second
	\item Third
    \end{itemize}
\end{frame}

\begin{frame}<1>[label=Cantor]
    \frametitle{againframe}

    \begin{Theorem}
	God doesn't exist.
    \end{Theorem}

    \begin{overprint}
	\onslide<1>
	\hyperlink{Cantor<2>}{\beamergotobutton{Proof}}

	\onslide<2->
	\begin{proof}
	    If god exist, why he/she never publish any important papers ?
	\end{proof}

	\hfill\hyperlink{Cantor<1>}{\beamerreturnbutton{Return}}
    \end{overprint}

\end{frame}

%%%%%%%%%%%%%%%%%%%%%%%%%%%%%%%%%%%%%%%%%%%%%%%%%%%%%%%%%%%%%%%%%%%%%%%%
\section{Advanced}

%%%%%%%%%%%%%%%%%%%%%%%%%%%%%%%%%%%%%%%%%%%%%%%%
\subsection*{Template}

\begin{frame}[fragile]
    \frametitle{Set beamertemplate}

    \begin{lstlisting}[language=TeX]
\setbeamertemplate{some beamer element} {your def. for this template}
    \end{lstlisting}

    options:
    \begin{itemize}
	\item [square] causes a small square to be used to render the template
	\item [circle]\{{\it radius}\} causes a circle of the given radius to ...
    \end{itemize}

    \begin{lstlisting}[language=TeX]
\setbeamertemplate{some beamer element} [square]
\setbeamertemplate{some beamer element} [circle]{3pt}
    \end{lstlisting}

\end{frame}

%%%%%%%%%%%%%%%%%%%%%%%%%%%%%%%%%%%%%%%%%%%%%%%%
\subsection*{Color}
\begin{frame}[fragile]
    \frametitle{Color}

    \begin{lstlisting}[language=TeX]
\setbeamercolor{some beamer element}{fg=red}
\setbeamercolor{some beamer element}{bg=black}
\setbeamercolor{some beamer element} {fg=red,bg=black}
    \end{lstlisting}

    \begin{itemize}
	\item structure color
	\item normal text color
    \end{itemize}

\end{frame}


%%%%%%%%%%%%%%%%%%%%%%%%%%%%%%%%%%%%%%%%%%%%%%%%
\subsection*{Font}
\begin{frame}[fragile]
    \frametitle{Font}
    \begin{lstlisting}[language=TeX]
\setbeamerfont{some beamer element} {family=\sffamily, series=\bfseries, shape=\itshape, size=\large}
    \end{lstlisting}
    Add a star to the command to first ``reset'' the font.
\end{frame}

\subsection{Template-Color-Font}
\begin{frame}[allowframebreaks]
    \frametitle{Beamer-Template/-Color/-Font}

    \begin{columns}[t]
	\column{0.5\textwidth}
	    \begin{itemize}
		\item head/foot
		    \begin{itemize}
			\item headline
			\item footline
			\item page number in head/foot
			\item section in head/foot
			\item subsection in head/foot
			\item subsubsection in head/foot
		    \end{itemize}
		\item sidebar
		    \begin{itemize}
			\item sidebar left
			\item sidebar right
			\item section in sidebar
			\item subsection in sidebar
			\item subsubsection in sidebar
		    \end{itemize}
		\item navigation
		    \begin{itemize}
			\item mini frame
			\item navigation symbols
		    \end{itemize}
		\item logo
		\end{itemize}

	    \column{0.5\textwidth}
		\begin{itemize}
		\item toc
		    \begin{itemize}
			\item section in toc
			\item section in toc shaded
			\item subsection in toc
			\item subsection in toc shaded
			\item subsubsection in toc
			\item subsubsection in toc shaded
		    \end{itemize}
		\item title
		    \begin{itemize}
			\item frametitle
			\item frametitle continuation
			\item title page
		    \end{itemize}
		\item background
		    \begin{itemize}
			\item background canvas
			\item background
		    \end{itemize}
		\item button
	    \end{itemize}
    \end{columns}

    \begin{columns}
	\column{0.5\textwidth}
	    \begin{itemize}
		\item itemize
		    \begin{itemize}
			\item itemize item
			\item itemize subitem
			\item itemize subsubitem
		    \end{itemize}
		\item enumerate
		    \begin{itemize}
			\item enumerate item
			\item enumerate subitem
			\item enumerate subsubitem
			\item enumerate mini template
		    \end{itemize}
		\item description
		    \begin{itemize}
			\item description item
		    \end{itemize}
		\item qed symbol
	    \end{itemize}
    \end{columns}
\end{frame}

\subsection*{Color-Font}
\begin{frame}
    \frametitle{Beamer-Color/-Font}

    \begin{description}
	\item[title]	{\bf title, author, institute, date}
    \end{description}
\end{frame}

\againframe<2>{Cantor}
\end{document}

