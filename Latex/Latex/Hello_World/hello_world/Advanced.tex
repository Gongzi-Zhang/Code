\chapter{Advanced}
\label{chap:Advanced}

%%%%%%%%%%%%%%%%%%%%%%%%%%%%%%%%%%%%%%%%%%%%%%%%%%%%%%%%%%%%%%%%%%%%%%%%
\section{Style}

%%%%%%%%%%%%%%%%%%%%%%%%%%%%%%%%%%%%%%%%%%%%%%%%
\subsection{Page Numbers}
How to change the numbering of pages:
\begin{lstlisting}{language=TeX}
\seccounter[page]{123}
\end{lstlisting}

How to remove all page numbers:
\begin{lstlisting}{language=TeX}
\pagestyle{empty}[page]{123}
\end{lstlisting}

\subsection{Citation}
How do I choose the square bracket or superscript style for citations:

%%%%%%%%%%%%%%%%%%%%%%%%%%%%%%%%%%%%%%%%%%%%%%%%
\subsection{Fonts}
\begin{lstlisting}[language=TeX]
\setmainfont{Times New Roman}	% serif fonts, for latin alphabetics
\setsansfont{helverica}	% latin non-serif alphabetics, usually for titles
\setmonofont{courier}	% same-width fonts, usually for code layout

% Chinese corresponding
\setCJKmainfont{simsun}
\setCJKsansfont{}
\setCJKmonofont{}   

% example: Linux Libertine
\setmainfont{LinLibertine\_R.otf}[
    BoldFont	= LinLibertine\_RZ.otf,
    ItalicFont	= LinLibertine\_RI.otf,
    BoldItalicFont = LinLibertine\_RZI.otf,
]
% sans-serif: Linux Biolinum
\setsansfont{LinLibertine\_R.otf}[
    BoldFont	= LinBiolimum\_RB.otf,
    ItalicFont	= LinBiolimum\_RI.otf,
    BoldItalicFont = LinBiolimum\_RBO.otf,
]
% typewriter type: Linux Libertine Mono
\setmonofont{LinLibertine\_M.otf}[
    BoldFont	= LinBiolimum\_MB.otf,
    ItalicFont	= LinBiolimum\_MI.otf,
    BoldItalicFont = LinBiolimum\_MBO.otf,
]

\setCJKmainfont[
    BoldFont	= Source Han Sans CN Medium,
    ItalicFont	= Adobe Kaiti Std R]
{Source Han Sans CN Light}
\setCJKsansfont[    % same as main
    BoldFont	= Source Han Sans CN Medium,
    ItalicFont	= Adobe Kaiti Std R]
{Source Han Sans CN Light}
\setCJKmonofont[
    BoldFont	= Source Han Sans CN Medium,
    ItalicFont	= Adobe Kaiti Std R]
{Source Han Sans CN Light}
\end{lstlisting}

%%%%%%%%%%%%%%%%%%%%%%%%%%%%%%%%%%%%%%%%%%%%%%%%%%%%%%%%%%%%%%%%%%%%%%%%
\section{Mode}
Using mode in \LaTeX{} allow one to choose different document class in one
tex file, for example:
\begin{lstlisting}[language=TeX]
\mode<presentation>{
    some preamble...
}

\mode<article>{
    preamble for article ...
}
\end{lstlisting}
Then if you compile a tex file (use the above one as input), then if the
documentclass is \textbf{beamer}, then mode \textbf{presentation} will be
used; if the documentclass is \textit{article}, then mode \textit{article}
will be choosed.

%%%%%%%%%%%%%%%%%%%%%%%%%%%%%%%%%%%%%%%%%%%%%%%%%%%%%%%%%%%%%%%%%%%%%%%%
\section{Adding note using tikz}
\newcommand{\annmark}[1]{%
    \textcolor{red}{$\langle$#1$\rangle$}%
}%

\newcommand{\ann}[1]{%
    \begin{tikzpicture}[remember picture, baseline=-0.75ex]%
        \node[coordinate] (inText) {};%
    \end{tikzpicture}%
    \marginpar{%
        \renewcommand{\baselinestretch}{1.0}%
        \begin{tikzpicture}[remember picture]%
            \definecolor{orange}{rgb}{1,0.5,0}%
            \draw node[fill=red!20,rounded corners,text width=\marginparwidth] (inNote){\footnotesize#1};%
    \end{tikzpicture}%
    }%
    \begin{tikzpicture}[remember picture, overlay]%
        \draw[draw = orange, thick]
            ([yshift=-0.2cm] inText)
                -| ([xshift=-0.2cm] inNote.west)
                -| (inNote.west);%
    \end{tikzpicture}%
}%

\setlength{\marginparwidth}{2.5cm}
\renewcommand{\baselinestretch}{1.3}

Freshness is the most important property for food \annmark{of course not 
for dry product }. And a good cooker will always keep food's freshness
and even enlarge the freshness using all kinds of methods.\ann{If one try to use spicy to hide all other flavor, then he must not a top cooker.} If you don't know how to cook a food, then the most obvious and simplest way is to boil it with water, which will sustain most of its freshness.
