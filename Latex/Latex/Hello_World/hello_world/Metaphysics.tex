\chapter{Fantacy in \LaTeX{}}

\section{Display}
When you find something abnormal about mathmetics (wrong color), check that
if there is blank \$\$ pair without anything in it.

%%%%%%%%%%%%%%%%%%%%%%%%%%%%%%%%%%%%%%%%%%%%%%%%%%%%%%%%%%%%%%%%%%%%%%%%
\section{Package}
Updating package, if you update your texlive, and then encounter some errors
that you never met before, then it is the problem of old-packages. Remember 
to update corresponding packages so that everything work properly. Especially
when you install a new system and a new texlive but import old personal configuration.

%%%%%%%%%%%%%%%%%%%%%%%%%%%%%%%%%%%%%%%%%%%%%%%%
In \LaTeX, you can load a package many times, but the option list of each 
package loading must be a subset of the options given at the first loading (exception
\package{fontenc})

However sometimes packages might be loaded in the document class already, or
there are constraints in the package order that prevents the reordering of the 
packages. Then \command|\PassOptionsToPackage| helps. It can even
be loaded before \command|\documentclass|. It gives the specified 
options to the package without loading the package.

Adding the options to the global options can also be a solution, but it is not the
best strategy, because also other unrelated packages see that options. Unknown global 
options are ignored by package, but known are then executed with unintended side effects.

\subsection{physics \& imakeidx}
If I put package imakeidx before physics, then the compiler will complain
something like this:
\begin{verbatim}
(/usr/share/texlive/texmf-dist/tex/latex/imakeidx/imakeidx.sty
(/usr/share/texlive/texmf-dist/tex/latex/xkeyval/xkeyval.sty
(/usr/share/texlive/texmf-dist/tex/generic/xkeyval/xkeyval.tex
(/usr/share/texlive/texmf-dist/tex/generic/xkeyval/xkvutils.tex)))
(/usr/share/texlive/texmf-dist/tex/latex/tools/multicol.sty))
(/home/weibin/texmf/tex/latex/physics/physics.sty
(/home/weibin/texmf/tex/latex/l3packages/xparse/xparse.sty
(/home/weibin/texmf/tex/latex/l3kernel/expl3.sty
(/home/weibin/texmf/tex/latex/l3kernel/expl3-code.tex
! Missing number, treated as zero.
<to be read again> 
                   \pdftex_shellescape:D 
l.23338	    }
\end{verbatim}
But if I put physics before imakeidx, then no error happen.

%%%%%%%%%%%%%%%%%%%%%%%%%%%%%%%%%%%%%%%%%%%%%%%%
\subsection{tikz \& graphicx}
It looks like \textbf{tika} package will load \emph{graphicx} package
automatically, so if you load \emph{graphicx} package manually with some
options, it will cause problem. For example, if I load them as
\begin{verbatim}
\usepackage[dvips]{graphicx}
\usepackage{tikz}
\end{verbatim}
it will result in:
\begin{verbatim}
Non-PDF special ignored!
\end{verbatim}
On the other hand, if I load them as:
\begin{verbatim}
\usepackage{tikz}
\usepackage[dvips]{graphicx}
\end{verbatim}
We get such error:
\begin{verbatim}
! LaTeX Error: Option clash for package graphicx.
\end{verbatim}

So this should be the conflicts between different options used in loading
\emph{graphicx} package. If you load only the \textbf{tikz} package, 
then everything works perfectly.

% graphics
! Paragraph ended before \\Gin@iii was complete.    \\
You will encounter this error if you load \emph{graphics} rather than \emph{graphicx}.
%%%%%%%%%%%%%%%%%%%%%%%%%%%%%%%%%%%%%%%%%%%%%%%%
\subsection{xcolor}
\package{xcolor} option error:
\begin{verbatim}
! LaTeX Error: Option clash for package xcolor.
\end{verbatim}
Package \package{pgf} also load \package{xcolor}, so
\package{xcolor} should be loaded before \package{pgf} or 
other packages that loads \package{pgf}.

%%%%%%%%%%%%%%%%%%%%%%%%%%%%%%%%%%%%%%%%%%%%%%%%%%%%%%%%%%%%%%%%%%%%%%%%
\section{Environment}

%%%%%%%%%%%%%%%%%%%%%%%%%%%%%%%%%%%%%%%%%%%%%%%%
\subsection{gathered}
No blank line allowed with \emph{gathered} environment, otherwise, it will show
error message:
\begin{verbatim}
! Missing $ inserted.
<inserted text> $
\end{verbatim}

%%%%%%%%%%%%%%%%%%%%%%%%%%%%%%%%%%%%%%%%%%%%%%%%%%%%%%%%%%%%%%%%%%%%%%%%
\section{Options}
%%%%%%%%%%%%%%%%%%%%%%%%%%%%%%%%%%%%%%%%%%%%%%%%
\subsection{aligned}
In the \emph{aligned} environment, if you begin you equation with \textbf{square bracket},
they will not be output normally, because aligned env. is set by the \emph{amsmath} package to scan ahead for a positioning augument such as [t] of [p]. Material that's found there but doesn't meet this format is simply discarded.	

The possible solutions are:
\begin{itemize}
    \item Insert \verb|\relax| before the left square bracket. It will stop the bracket from being interpreted as an argument.
    \item Insert \{\} before the left square brackedt.
\end{itemize}
