\documentclass{ctexrep}
% \documentclass{ctexart}
% \usepackage{ctex}
\usepackage{booktabs}
\usepackage{enumitem}
\usepackage[colorlinks]{hyperref}
    \hypersetup{
	colorlinks = ture,
	linkcolor = black,
	urlcolor = blue,
    }
\usepackage{listings}
\usepackage{tikz}
    \usetikzlibrary{arrows,shapes,chains,positioning}
\graphicspath{{figures/}}

\setCJKmainfont{WenQuanYi Micro Hei}	% 设置字体

\title{中文\TeX{}简介}
\author{张卫斌}

\begin{document}
\maketitle
\tableofcontents
\chapter{简介}
你好吗?

\section{文件类}
可用的文件类有: ctexart, ctexrep, ctexbook.
即只需在原类的基础上加上ctex,同时缩写类名即可。    \\
\textbf{注意:并没有ctexletter类型}。

\subsection{安装包}
为了适用 ctexart, ctexrep, ctexbook 等文件类,我们需要安装 
{\color{red} texlive-lang-chinese} 包以安装相关的类文件: ctexart.cls ...

\subsection{字体}
中文字体可用文泉驿字体(http://wenq.org/wqy2/index.cgi?Home), 下载选中字体,
解压缩后将字体文件(*.ttc) 放置在 /usr/share/fonts/wenquanyi 目录下即可。

\section{编译}
中文\LaTeX{} 文档需用 xelatex 编译。

\include{fonts}
\end{document}
