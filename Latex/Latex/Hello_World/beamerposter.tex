\PassOptionsToPackage{dvipsnames}{xcolor}
\documentclass{beamer}
\usepackage[english]{babel}
\usepackage[dvipsnames]{xcolor}
\usepackage[orientation=portrait, size=a0, scale=1.4]{beamerposter}
\usetheme{Berlin}

\title{Beamer Poster}
\author{ShareLaTeX Team}
\institute{ShareLaTeX Ins.}
\date{\today}
% \logo{\includegraphics{logo}}	% this command won't work in most of the
% themes, and has to be set by hand in the themes's.sty file.

\begin{document}
\begin{frame}{}
    \maketitle
    \vfill
    \begin{block}{Introduction}
	The package \textbf{beamerposter} enhances the capabilities of 
	the \textcolor{red}{standard beamer class}, making it possible
	to create scientific posters with the same syntax of a beamer 
	presentation.

	By now there are not many themes for this package, and it is
	slightly less flexible than \textbf{tikzposter}, but if you
	are familiar with beamer, using \textbf{beamerposter} don't
	require learning new commands.
    \end{block}
    \begin{block}{}
	Since the document class is \textbf{beamer}, to create the poster
	all the contents must be typed inside a frame environment.
    \end{block}
    \begin{block}{preamble}
	poster size: a0, a1, a2, a3 and a4, but the dimensions can be 
	arbitrarilly set with the options \textcolor{RawSienna}{\textbf{width=x, height=y}}. \\
	scale: scale the fonts
    \end{block}
    \begin{columns}[t]
	\begin{column}{0.35\linewidth}
	    \begin{block}{Column1}
		\begin{itemize}
		    \item first
		    \item second
		\end{itemize}
	    \end{block}
	\end{column}
	\begin{column}{0.55\linewidth}
	    \begin{block}{Column2}
		\begin{itemize}
		    \item third
		    \item fourth
		\end{itemize}
	    \end{block}
	\end{column}
    \end{columns}
\end{frame}
\end{document}
