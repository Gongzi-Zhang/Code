\section{Basic}
Basic ideas and concepts in \G{}.

\begin{description}
    \item [Step]
	Sample NMFP (Number of Mean Free Path) by the \emph{material independent way}	$\Rightarrow$ Using the xsection in the material where the particle is currently in, converts NMFP to the PL (physical length) $\Rightarrow$ The process which has the minimum PL determines the step length.
\end{description}
\subsection{Example}
\begin{enumerate}
    \item General
	\begin{enumerate}
	    \item Configure the \textbf{Run}
	    \item Configure the \textbf{Event} Loop
	\end{enumerate}
    \item Experimental set-up
	\begin{enumerate}
	    \item geometrical set-up.
	    \item the coordinates of impact of tracks in the layers of the
		tracker, energy release in the strips of the tracker.
	    \item energy deposited in calorimeter
	    \item energy deposited in anticoincidence(?)
	    \item Digitise the hits, setting a threshold for the energy
		deposit in the tracker
	    \item Generate a trigger signal combining signals from different
		detectors.
	\end{enumerate}
    \item Physics
	\begin{enumerate}
	    \item Primary events
	    \item Electromagnetic processes appropriate to the energy range
		of the experiment
	    \item Hadronic processes
	\end{enumerate}
    \item Analysis
	\begin{enumerate}
	    \item x-y distribution of impact of the track
	    \item histograms during the simulation execution
	    \item store significant quantities in a ntuple
	    \item Plot energy distribution in the calorimeter
	\end{enumerate}
    \item Visualisation
	\begin{enumerate}
	    \item Visualize the experimental set-up
	    \item trakcs
	    \item hits
	\end{enumerate}
    \item UI
	\begin{enumerate}
	    \item Configure the tracker, by modifying the number of active
		planes, the pitch of the strips, the area of silicon tiles,
		the material of the converter.
	    \item Configure the calorimeter, by modifying the number of
		active elements, the number of layers.
	    \item Sources
	    \item Digisation by modifying threshold
	    \item Histograms
	\end{enumerate}
    \item Persistency
	\begin{enumerate}
	    \item Produce an intermediate output of the simulation at the
		level of hits in the tracker
	    \item Store significant results in FITS format
	    \item Read in an intermediate output for further elaboration.
	\end{enumerate}
\end{enumerate}
