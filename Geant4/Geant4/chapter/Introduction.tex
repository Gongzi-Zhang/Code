\section{Introduction}
This is a personal introduction to \G{}.
https://www.ge.infn.it/geant4/events/nss2003/geant4lectures.html

Event biasing (variance reduction) techniques:
\begin{itemize}
    \item Primary event biasing	\\
	Biasing primary events/particles in terms of type of event, momentum
	distribution. 
    \item Leading particle biasing \\
	Taking only the most energetic (or most important) secondary. \\
	Simulating of a full shower is an expensive calculation. Instread of
	generating a full shower, trace only the most energetic secondary.
	Other secondary particles are immediately killed befoer being
	stacked. In this way, it is convenient to roughly estimate, e.g. the
	thickness of a shielf. Of course, physical quantities such as energy
	are not conserved for each event.
    \item Physics based biasing \\
	Biasing secondary production in terms of particle type, momentum
	distribution, cross-section, etc.
    \item Geometry based biasing \\
	Importance weighting for volume/region. \\
	Duplication of sudden death of tracks.	\\
	Define importance for each geometrical region, duplicate a track
	with relative weight if it goes toward more important region.
	Russian-roulette in another direction.
	Scoring particle flux with weights.
    \item Forced interaction \\
	Force a particular interaction, e.g. within a volume.
\end{itemize}
